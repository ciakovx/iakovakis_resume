%%%%%%%%%%%%%%%%%%%%%%%%%%%%%%%%%%%%%%%%%
% Medium Length Professional CV
% LaTeX Template
% Version 2.0 (8/5/13)
%
% This template has been downloaded from:
% http://www.LaTeXTemplates.com
%
% Original author:
% Trey Hunner (http://www.treyhunner.com/)
%
% Important note:
% This template requires the resume.cls file to be in the same directory as the
% .tex file. The resume.cls file provides the resume style used for structuring the
% document.
%
%%%%%%%%%%%%%%%%%%%%%%%%%%%%%%%%%%%%%%%%%

%----------------------------------------------------------------------------------------
%	PACKAGES AND OTHER DOCUMENT CONFIGURATIONS
%----------------------------------------------------------------------------------------

\documentclass{resume} % Use the custom resume.cls style

\usepackage[left=.5in,top=1in,right=.5in,bottom=1in]{geometry} % Document margins
\usepackage{marvosym}

\name{Clarke L. Iakovakis} % Your name
\address{11422 Derby Ct. \\ Perkins, OK 74059} % Your address
\address{\Telefon~(405)~744$\cdot$9743  \\ \Letter~clarke.iakovakis@okstate.edu} % Your phone number and email

\usepackage{hyperref}


\hypersetup{%
        colorlinks=true,
        citecolor={green},
        linkcolor={red},
        urlcolor={blue}}

\begin{document}

%----------------------------------------------------------------------------------------
%	EDUCATION SECTION
%----------------------------------------------------------------------------------------

\begin{rSection}{Education}
\raggedright
{\bf University of Houston-Clear Lake} \hfill{Coursework complete--Degree unfinished} \\
\textit{Master of Arts, History}
\begin{rSubsectionEd}
	\item GPA: 3.8
\end{rSubsectionEd}
{\bf University of Texas at Austin} \hfill {May 2011} \\ 
\textit{Master of Science, Information Studies}
\begin{rSubsectionEd}
	\item Awarded full tuition scholarship from IMLS
	\item GPA: 4.0
	\item Thesis: \href{http://hdl.handle.net/2152/ETD-UT-2011-05-3522}{``An interdisciplinary inquiry into the ethics codes of the helping professions : interpretations of moral principles and professional responsibilities''} \\	
\end{rSubsectionEd}

{\bf Texas State University} \hfill {May 2006} \\ 
\textit{Bachelor of Arts, History} with Minor, Writing 

\end{rSection}

%----------------------------------------------------------------------------------------
%	LIBRARY EXPERIENCE SECTION
%----------------------------------------------------------------------------------------

\begin{rSection}{Library Experience}
\raggedright

\begin{rSubsection}{Oklahoma State University}{Edmon Low Library}{August 2018-Present}{Scholarly Services Librarian}{Stillwater, OK}
\item Lead implementation of Symplectic Elements Research Information Management System, including project management, data mapping, training, data migration, user support, and all aspects related to implementation.
\item Provide leadership for the Scholarly Services \& Research Engagement Department, including supervision of the Data Initiatives and the Digital Scholarship Librarians
\item Create advocacy programs that encourage OSU faculty and students to adopt open access research and publishing practices, including training workshops, consultations, and more
\item Provide workshops and instruction on data and computing technologies including R, OpenRefine, Git, Open Science Framework, \LaTeX, and others. Serve as Lead Maintainer for the Library Carpentry R Session. Help organize and lead the OSU Carpentry workshops. 
\item Develop partnerships with internal and external constituents to advance scholarly services and initiatives such as advocating for author rights, developing a model for copyright and Creative Commons license support, raising awareness about open access best practices including depositing works into the institutional repository, and managing scholarly impact profiles and analytics.
\item Undertake academic department liaison responsibilities for the History and Political Science programs
\end{rSubsection}

\pagebreak

\begin{rSubsection}{University of Houston-Clear Lake}{Alfred R. Neumann Library}{January 2017 - August 2018}{Scholarly Communications Librarian}{Houston, TX}
\item Launched the university's first scholarly communications program
\item Assumed formal management of format \& submission of all university theses \& dissertations, supervising one library employee, administering of Vireo \& DSpace, and liaising to all deans and advising coordinators
\item Established working relationships with individual faculty members \& faculty evaluation committees, holding individual consultations for journal evaluation \& hosting workshops on topics such as ``Measuring Your Research Impact," ``Understanding Altmetrics," \& ``Using the Publish or Perish Software"
\item Collaborated with technical services \& research librarians to wrangle collections data \& developed interactive visualizations in order to make informed purchasing decisions
\item Conducted a three-part workshop series for Open Access Week 2017 on Open Access, Intellectual Property, \& Open Data 
\item Developed \& hosted workshops for students on using EndNote, R, and the Word templates I created for theses \& dissertations
\item Co-chaired the committee to host \href{http://clearlake2017.thatcamp.org/}{THATCamp Clear Lake}, November 2017
\item Continued to maintain full liaison responsibilities in four subject areas \& staff the reference desk 5-10 hours per week
\end{rSubsection}


\begin{rSubsection}{University of Houston-Clear Lake}{Alfred R. Neumann Library}{September 2014 - January 2017}{Research \& Instruction Librarian}{Houston, TX}
\item Initiated and led transition from print to electronic theses \& dissertations, including project management, presentations to university administration \& faculty, writing formal \href{http://uhcl.libguides.com/dissertation}{format \& submission procedures}, \href{http://uhcl.libguides.com/thesesdissertations}{outreach to faculty}. 
\item Managed the inventory \& assessment of the 10,000 item K-12 Curriculum Library, administering survey to faculty, establishing a Faculty Advisory Committee, purchasing updated and current textbooks and teaching materials, and reorganizing the entire space to facilitate browsing and retrieval
\item Partnered with technical services staff to streamline library data management \& cleaning, such as data merging, string editing, and other data manipulation
\item Lead and actively participated in several library \& university committees--such as the Vireo Working Group, Digital Library Committee, and the University Life Committee
\item Served as liaison to Criminology, Curriculum \& Instruction, Instructional Technology, \& Geography, performing all responsibilities including reference service, instruction, collection development, accreditation support, \& documenting library resources to support new programs
\item Developed curriculum \& delivered library instruction to all levels of students, and to humanities, social sciences, and science courses. View a \href{https://www.slideshare.net/ciakov/presentations}{sample of instruction slides}
\end{rSubsection}

\pagebreak

\begin{rSubsection}{University of Texas at Arlington}{UT Arlington Libraries}{August 2013 - August 2014}{Data \& E-Science Librarian}{Arlington, TX}
\item Developed and presented instruction, workshops and consultations with faculty, students, \& library staff on range of library topics, including scholarly communications \& open access, data reference \& analysis tools, 3D printing, research data management \& compliance with federal data management plans
\item Team Leader, 2013 CLIR-DLF E-Science Institute: Initiated and led an assessment of university research practices, interviewing 18 university administrators, deans, and faculty members, transcribing the interviews myself, and conducting qualitative analysis to develop a SWOT Analysis
\item Organized and hosted GIS Day in 2012 \& 2013, and the biannual Cross Timbers Library Collaborative Scholarly Communications \& Digital Curation meeting in 2013
\item Served as liaison to physics, biology \& computer science departments, performing reference consultations with students and faculty, teaching information literacy courses, \& developing and maintaining online research guides

\end{rSubsection}

%------------------------------------------------

\begin{rSubsection}{University of Texas at Arlington}{UT Arlington Libraries}{August 2011 - August 2013}{Reference  \& Information Literacy Assistant}{Arlington, TX}
\item Developed curriculum \& taught information literacy classes for several graduate \& undergraduate classes, including performing general and specific library research, citation formatting, \& avoiding plagiarism
\item Researched, designed, and launched online guides to using e-books and conducting mobile research
\item Supervised reference desk in evenings, acting as manager of entire reference service, answering all student questions, and enforcing library policies and procedures
\item Created and maintained wiki for department of Information Services
\item Comprehensive knowledge of LoC, Dewey, and SuDoc classification
\item Assisted the nursing librarian with a two-year study examining the impact of embedding a librarian on student performance in online courses
\item Researched, wrote, and published two articles in the \textit{Arlington Citizen-Journal} newspaper, based on research of historical photographs from the library's special collections
\end{rSubsection}

%------------------------------------------------

\begin{rSubsection}{University of Texas at Austin}{Nettie Lee Benson Latin American Collection}{August 2009 - January 2011}{Circulation Associate}{Austin, TX}
\item Explained library policies, answered informational and directional questions, and helped patrons use and understand the online library catalog
\item Performed range of circulation tasks: interacting with users, checking items in and out, placing reserves, and changing item statuses
\end{rSubsection}

%------------------------------------------------
\pagebreak

\begin{rSubsection}{Austin Public Library}{Will Hampton Branch at Oak Hill}{October 2007 - August 2009}{Administrative Associate}{Austin, TX}
\item Provided general information services to a diverse patron base: answered reference and reader’s advisory questions, and provided library and online catalog instruction
\item Organized and updated binder of suggested and popular reading materials for patron perusal
\item Sorted and shelved all types of library materials, and designed and executed shelf reorganization projects when necessary
\end{rSubsection}



\end{rSection}

\pagebreak
%----------------------------------------------------------------------------------------
%	CONFERENCE PRESENTATIONS
%----------------------------------------------------------------------------------------

\begin{rSection}{Publications \& Conference Presentations}

\begin{rSubsectionConf}{2em}
	
Macken, M. and Iakovakis, C. “Privacy and Research Information Management Systems.” \textit{Serials Librarian}. [accepted for publication in 2021 special issue on libraries \& privacy].

Iakovakis, C., Essmiller, K., and Upson, M. “Unspoiled broth: A Memorandum of Understanding for chefs cooking up OER.” \textit{The Scholarly Communications Cookbook}. Association of College \& Research Libraries: Chicago. [accepted for publication in 2021].

Essmiller, K., Iakovakis, C., and Upson, M. “The HackYourSyllabus Mini-Grant: A Bite-Sized OER Incentive Program. \textit{The Scholarly Communications Cookbook}. Association of College \& Research Libraries: Chicago. [accepted for publication in 2021].

Iakovakis, C. ``Working with Scholarly Literature in R: Pulling, Wrangling, Cleaning, and Analyzing Structured Bibliographic Metadata.''  FORCE11 Scholarly Communications Institute. August 2020. \href{https://www.force11.org/fsci/2020/fsci-2020-course-abstracts#W24}{Session Description}. \href{https://ciakovx.github.io/fsci_syllabus.html}{Session Materials}. \href{https://osf.io/ubd9r/}{Session Recordings}. 
 
Iakovakis, C. Feeney, P., and Burke, S. ``Using the Crossref API: Pulling and working with Crossref publication and funder metadata in R.'' Pre-conference workshop for Electronic Resources \& Libraries conference. March 2020. \href{https://sched.co/XVhQ}{Session Description}. \href{https://osf.io/tz246/}{Session Materials}.

Iakovakis, C. ``Building Your Online Research Profile.” Coalition for Advancing Digital Research \& Education (CADRE) Conference. April 2020. \href{https://cadre.library.okstate.edu/data/presentations/2020/building-your-online-research-profile}{Session Description}. \href{https://cadre.library.okstate.edu/sub-domains/cadre/presentations/Iakovakis_Building_Profile_CADRE.pdf}{Session Materials}.

Lampert, D., Prud’homme, P.A., Doehle, P. and Iakovakis, C. ``Did it Work? Strategies for Assessing a Digital Education Program.” Coalition for Advancing Digital Research \& Education (CADRE) Conference. April 2020. href{https://cadre.library.okstate.edu/data/presentations/2020/did-it-work-strategies-for-assessing-a-digital-education-program}{Session Description}. \href{https://cadre.library.okstate.edu/sub-domains/cadre/presentations/doehle_diditwork.pptx}{Session Materials}. \href{https://osf.io/vj72t/}{Session Recording}.

Iakovakis, C. and Burkholder, K. “Copyright and Distance Education.” Oklahoma Library Association Conference. April 2020. \href{https://www.oklibs.org/page/OLA2020_24_Friday}{Session Description}.

Iakovakis, C. ``Working with Scholarly Literature in R: Pulling, Wrangling, Cleaning, and Analyzing Structured Bibliographic Metadata.'' FORCE11 Scholarly Communications Institute. August 2019. \href{https://www.force11.org/fsci/2019/course-abstracts#AM4}{Session Description}. \href{https://ciakovx.github.io/fsci_syllabus.html}{Session Materials}
 
Iakovakis, C. ``Querying \& Accessing Scholarly Literature metadata: Using rcrossref, rorcid, and roadoi.'' Texas Conference on Digital Libraries. May 2019. \href{https://www.tdl.org/tcdl-2019/tcdl-2019-pre-conference-registration/}{Session Description}. \href{https://tdl-ir.tdl.org/handle/2249.1/156426}{Session Materials}.

Iakovakis, C. and Burke, S. ``Data as a Core Competency: Wrangling Library Data and Building Campus Connections.'' Poster. ACRL 2019. April 2019. \href{http://s4.goeshow.com/acrl/national/2019/profile.cfm?profile_name=session&master_key=681FAAEE-D282-252F-ED91-91AC04A95FCD&page_key=126CB9A0-B53A-2A1E-9827-DF2938A26C80&xtemplate&userLGNKEY=0}{Session Description}. \href{https://acrl2019-acrl.ipostersessions.com/default.aspx?s=04-FE-E6-99-F3-96-80-C4-1A-16-1C-00-A7-39-6F-2A}{Poster}
 
Iakovakis, C. and Burke, S. ``Confronting the Elephant in the Room: Cleaning \& Wrangling Data for Collections \& Scholarly Services.'' Electronic Resources \& Libraries conference. March 2019. \href{https://erl19.sched.com/event/I8Cx/s54-confronting-the-elephant-in-the-room-cleaning-and-wrangling-data-for-collections-and-scholarly-services}{Session Description}. \href{https://osf.io/a5p3r/}{Session Materials}.

Muschelli, J. and Iakovakis, C. ``Gathering Multiple Subject Queries with Large Numbers of Authors.” Vignette for rscopus R package. December 2018. \url{http://johnmuschelli.com/rscopus/articles/multi_author.html}

Iakovakis, C. ``Centralizing ETD Processing in the Library: Owning the Roles of ETD Administration, Research Librarianship, and Scholarly Communications." US Electronic Theses \& Dissertation Association Annual Conference. October 2018. \href{http://www.ocs.usetda.org/index.php/USETDA/USETDA2018/paper/view/138}{Session Description}.

Iakovakis, C. ``Introduction to R for Libraries." Three-part webinar. Sponsored by the Association for Library Collections \& Technical Services (ALCTS). May 2018. \href{http://www.ala.org/alcts/confevents/upcoming/webinar/IntrotoR}{Session Description}.

Iakovakis, C. and Burke, S. ``Database Usage in Context: Wrangling Vendor, Library, and Institutional Data." Webinar at the Amigos Library Services Online Conference \textit{Wrangling Library Data: Analytics, Dashboards, and Spreadsheets}. February 2018. \href{http://www.amigos.org/wrangling_data}{Session Description}. \href{http://uhcl.libguides.com/alcts/amigos}{Session Materials}.

Iakovakis, C. and Burke, S. ``Personalizing Assessment: Making Collections Data Work for You." Panel presentation at the American Library Association Annual Conference.  Chicago, IL. June 2017. \href{https://alcts.ala.org/news/2017/ac-personalizing-assessment/}{Session Description}. \href{http://uhcl.libguides.com/alcts}{Session Materials}.

Iakovakis, C. and Burke, S. ``Revitalizing the Curriculum Library: Analyzing the Collection, Surveying Faculty, and Forming a College of Education Advisory Group." Lightning talk co-presented at the Education and Behavioral Sciences Section (EBSS) Research Forum at the ALA Annual Conference. Sponsored by EBSS. Chicago, IL. June 2017. \href{http://www.ala.org/acrl/sites/ala.org.acrl/files/content/aboutacrl/directoryofleadership/sections/ebss/ebsswebsite/ebssnewsletter/ebss_fall2017.pdf}{Session Description (p. 17)}. \href{http://www.ala.org/acrl/aboutacrl/directoryofleadership/sections/ebss/ebsswebsite/poster-forum}{Session Materials}. 

Ford, L., Holland, J. and Iakovakis, C. ``I Don’t Know What I’m Looking At: Understanding Student Libguide Use with Eye-Tracking Software.” Paper co-presented at the Association of College \& Research Libraries Biennial Conference. Baltimore, MD. March 2017. \href{http://www.ala.org/acrl/sites/ala.org.acrl/files/content/conferences/confsandpreconfs/2017/IDontKnowWhatImLookingat.pdf}{Conference Paper}.

Mirza, R. and Iakovakis, C. ``Developing Library Services for Digital Humanities \& E-Science Support Using Qualitative Research." Co-presented at Digital Frontiers. Denton, TX. September 2014. \href{https://rafiamirzasite.wordpress.com/2016/03/13/developing-library-services-for-digital-humanities/}{Session Description}.\href{https://www.slideshare.net/librarianrafia/digital-frontiers-2014}{Session materials}
\end{rSubsectionConf}

I have also presented at multiple state and local conferences, including Oklahoma ACRL (2018), Texas Library Association District 8 Meeting (2015-2017), IEEE Dual Conference of Innovation \& Automation (2017), THATCamp Clear Lake (2017), UHCL GIS Day (2015), and the Cross Timbers Library Collaborative (2013).


\end{rSection}

\pagebreak
%----------------------------------------------------------------------------------------
%	PROFESSIONAL DEVELOPMENT
%----------------------------------------------------------------------------------------

\begin{rSection}{Professional Service}

\begin{tabular}{ @{} >{\bfseries}l @{\hspace{2ex}} l }
OSU Service & Head, Scholarly Services \& Research Engagement Department \\
	& Member, Faculty Council Student Affairs \& Learning Committee, 2019-present \\
	& Member, Library Collection Development Committee, 2018-present \\
	& Secretary, Library Faculty Council, 2019-present \\
	& Hiring Committee member: Digital Services Librarian \\
	& Hiring Committee member: Asst. Dir. McNair Scholars Program \\
State/Ntl. Service & Board Member, Oklahoma Association of College \& Research Libraries, 2019-present \\
	& Planning Committee Chair, Texas Conference on Digital Libraries, 2018-19 \\
	& Board Member, Texas Electronic Thesis \& Dissertation Association, 2017-19 \\
Conferences & Oklahoma Association of College \& Research Libraries Conference, 2018-19 \\
	& Digital Science User Day, 2019 \\
	& Force11 Scholarly Communications Institute, 2019 \\
	& Electronic Resources \& Libraries, 2019 \\
	& Texas Conference on Digital Libraries, 2016-2019 \\
	 & TLA Annual Conference, 2008-2012, 2015-16 \\
	 & ALA Annual Conference, 2009, 2017 \\
	 & ACRL Biennial Conference, 2017 \\
	 & THATCamp Clear Lake (co-chair), 2017 \\
	 & US ETDA Annual Conference, 2016, 2018 \\
	 & CLIR-DLF E-Science Institute Capstone, April 2013 \\
	 & ACRL Scholarly Communications Workshop, 2013-14, 2017 \\
	 & ICPSR Biennial Meeting, October 2013 \\
Coursework & Digital Libraries, Reference \& Instruction, \\
 & Instructional Psychology, Information Seeking Behavior, \\
 & Collection Management, Information Policy, \\
 & Cataloging \& Metadata \\
Research & Scholarly Communication, Open Access Publishing, \\
 & Data Science, Collection Analysis, \\
 & Information Literacy Instructional Methodology \\
 & Post-WWII American political history \& 20\textsuperscript{th} Century literature

\end{tabular}

\end{rSection}


%----------------------------------------------------------------------------------------
%	TECHNICAL STRENGTHS SECTION
%----------------------------------------------------------------------------------------

\begin{rSection}{Technical Strengths}

\begin{tabular}{ @{} >{\bfseries}l @{\hspace{2ex}} l }
Certified Carpentries Instructor & 2019-present \\
Word Processing & Microsoft Word, \LaTeX, Markdown \\
Design & Adobe Creative Suite (Illustrator, Acrobat \& Photoshop), \\
  & Google Sketchup, Microsoft Publisher \\
Data \& Statistical analysis & R, Tableau, Microsoft Excel, Microsoft Access, OpenRefine \\
Tools & GIS (ArcMap), \href{https://github.com/ciakovx}{Git}, Open Science Framework \\
Web & CSS, HTML, Adobe Dreamweaver \\
Integrated Library Systems & Sierra, Voyager, Millenium \\
\end{tabular}

\end{rSection}


\end{document}
